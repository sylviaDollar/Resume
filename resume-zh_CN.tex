% !TEX TS-program = xelatex
% !TEX encoding = UTF-8 Unicode
% !Mode:: "TeX:UTF-8"

\documentclass{resume}
\usepackage[colorlinks,
            linkcolor=red,
            anchorcolor=blue,
            citecolor=green
            ]{hyperref}
\usepackage{zh_CN-Adobefonts_external} % Simplified Chinese Support using external fonts (./fonts/zh_CN-Adobe/)
%\usepackage{zh_CN-Adobefonts_internal} % Simplified Chinese Support using system fonts
\usepackage{linespacing_fix} % disable extra space before next section
\usepackage{cite}

\begin{document}
\pagenumbering{gobble} % suppress displaying page number

\name{姜楠}

% {E-mail}{mobilephone}{homepage}
% be careful of _ in emaill address
\contactInfo{nanjiang@buaa.edu.cn}{(+86) 131-4651-9692}{\href{http://github.com/jiangnanHugo}{github.com/jiangnanHugo}}
% {E-mail}{mobilephone}
% keep the last empty braces!
%\contactInfo{xxx@yuanbin.me}{(+86) 131-221-87xxx}{}

\section{\faGraduationCap\  教育背景}
\datedsubsection{\textbf{北京航空航天大学}}{2015 - 至今}
\textit{在读硕士研究生}\ 计算机科学与技术专业 \\
计算机学院
\datedsubsection{\textbf{浙江工业大学}}{2011 - 2015}
\textit{学士}\ 计算机科学与技术 + 自动化专业(双专业)\\
计算机科学与技术学院 \\
平均绩点: 4.0, 排名: 1/60

\section{\faLightbulbO\ 研究方向}
自然语言处理,机器学习,深度学习,神经网络

\section{\faBook\ 已发表论文}
\begin{itemize}[parsep=0.5ex]
\item Jiang N, Rong W, Peng B, et al. Modeling Joint Representation with Tri-Modal Deep Belief Networks for Query and Question Matching[J]. IEICE Transactions, 2016, 99(4): 927-935.

\item Jiang N, Rong W, Peng B, et al. An empirical analysis of different sparse penalties for autoencoder in unsupervised feature learning[C]//Neural Networks (IJCNN), 2015 International Joint Conference on. IEEE, 2015: 1-8.
\end{itemize}

\section{\faUsers\ 实习/项目经历}
\datedsubsection{\textbf{教育部工程中心}}{2015.10}
\role{北京航空航天大学}{研究助理}
大四下学期进入该实验室实习,学习机器学习算法,主要研究NLP方向.期间,发表一篇会议论文、一篇期刊论文,参加一次国际会议,获得该会议颁发的Travel Grant.


\datedsubsection{\textbf{图形图像实验室}}{2013.9}
\role{浙江工业大学}{iOS/Android 开发员}
学习Android/iOS 基本框架,开发后台和手机后台数据通信的基本框架,并将数据以各种图表形式绘制在手机页面上,并利用此项目申请了国家创新科技立项。



% Reference Test
%\datedsubsection{\textbf{Paper Title\cite{zaharia2012resilient}}}{May. 2015}
%An xxx optimized for xxx\cite{verma2015large}
%\begin{itemize}
%  \item main contribution
%\end{itemize}

\section{\faCogs\ IT 技能}
% increase linespacing [parsep=0.5ex]
\begin{itemize}[parsep=0.5ex]
  \item 编程语言: Python, C/C++, Java, R, Bash
  \item 工具: Linux, Git, Vim, Matlab
  \item 开发库: Theano, Keras, Numpy, Tensorflow
\end{itemize}

\section{\faStar\ 获奖情况}
\datedline{二等奖学金(北京航空航天大学)}{2015}
\datedline{Travel Grant of IEEE IJCNN(IEEE)}{2015}
\datedline{优秀毕业生(浙江工业大学)}{2015}
\datedline{校一等奖学金(浙江工业大学)}{2014,2013,2012}
\datedline{国家励志奖学金(浙江工业大学)}{2013}
\datedline{国家奖学金(浙江工业大学)}{2012}


\section{\faInfo\ 语言水平}
% increase linespacing [parsep=0.5ex]
\begin{itemize}[parsep=0.5ex]
  \item 英语考试: CET-4(602), CET-6(566)
  \item 写作水平: 熟练
\end{itemize}

%% Reference
%\newpage
%\bibliographystyle{IEEETran}
%\bibliography{mycite}
\end{document}
