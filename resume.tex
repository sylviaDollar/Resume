% !TEX program = xelatex

\documentclass{resume}
%\usepackage{zh_CN-Adobefonts_external} % Simplified Chinese Support using external fonts (./fonts/zh_CN-Adobe/)
%\usepackage{zh_CN-Adobefonts_internal} % Simplified Chinese Support using system fonts

\begin{document}
\pagenumbering{gobble} % suppress displaying page number

\name{Nan Jiang}

% {E-mail}{mobilephone}{homepage}
% be careful of _ in emaill address
\contactInfo{nanjiang@buaa.edu.cn}{(+86) 131-4651-9692}{github.com/jiangnanHugo}
% {E-mail}{mobilephone}
% keep the last empty braces!
%\contactInfo{xxx@yuanbin.me}{(+86) 131-221-87xxx}{}

\section{\faGraduationCap\ Education}
\datedsubsection{\textbf{Beihang University}}{\emph{Sept. 2015}}
\textit{Master student} in Computer Science \\
School of Computer Science and Engineering
\datedsubsection{\textbf{Zhejiang University of Technology}}{\emph{Sept. 2011}}
\textit{B.S.} in Computer Science \& Automation(double degree)\\
College of Computer Science and Technology\\
Major GPA: 4.0, Top 1 of 60

\section{\faLightbulbO\ Research Interests}
Bioinformatics, Sentiment Analysis, Natural Language Processing.

Neural Network, Deep Learning, Machine Learning.

\section{\faBook\ Publications}
\begin{itemize}[parsep=0.5ex]
\item Jiang N, Rong W, Peng B, et al. Modeling Joint Representation with Tri-Modal Deep Belief Networks for Query and Question Matching[J]. IEICE Transactions, 2016, 99(4): 927-935.

\item Jiang N, Rong W, Peng B, et al. An empirical analysis of different sparse penalties for autoencoder in unsupervised feature learning[C]//Neural Networks (IJCNN), 2015 International Joint Conference on. IEEE, 2015: 1-8.
\end{itemize}
\section{\faUsers\ Experiences}
\datedsubsection{\textbf{Engineering Research Center of Ministry of Education}}{Sept. 2015}
\role{Research Assistant}{}
Learning the basics of machine learning and study the fields on bioinformatics.


\datedsubsection{\textbf{Laboratory of Graphics and Images}}{July. 2013}
\role{iOS/Android Developer}{}
Develop app for iOS/Android system for demonstrating data charts, and write the modules on communicating with server.


% Reference Test
%\datedsubsection{\textbf{Paper Title\cite{zaharia2012resilient}}}{May. 2015}
%An xxx optimized for xxx\cite{verma2015large}
%\begin{itemize}
%  \item main contribution
%\end{itemize}

\section{\faCogs\ Technical Strengths}
\begin{itemize}[parsep=0.5ex]
  \item Languages: Python, C/C++, Java, R, Bash, Matlab
  \item Tools: Linux, Git, Vim, Sublime
  \item Library: Theano, Keras, Numpy, Tensorflow
\end{itemize}

\section{\faStar\ Honors and Awards}
\datedline{The Second Prize Scholarship(Beihang University)}{2015}
\datedline{Travel Grant of IEEE IJCNN(IEEE)}{2015}
\datedline{Outstanding Graduate Student(Zhejiang University of Technology)}{2015}
\datedline{The First Prize Scholarship(Zhejiang University of Technology)}{2014,2013,2012}
\datedline{National Scholarship for Encouragement(Zhejiang University of Technology)}{2013}
\datedline{National Scholarship(Zhejiang University of Technology)}{2012}

\section{\faGlobe\ Language Qualifications}
\begin{itemize}[parsep=0.5ex]
  \item Chinese: Native;
  \item English: Fluent(CET-6:566);
\end{itemize}

%% Reference
%\newpage
%\bibliographystyle{IEEETran}
%\bibliography{mycite}
\end{document}
